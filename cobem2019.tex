% LaTeX .tex
% Example for the proceedings of the  25th International Congress of Mechanical Engineering
% COBEM 2019
% October, 20-25, 2019, Uberlândia, MG, Brazil
% Based on the template of the proceedings of COBEM2015 and COBEM2017

\documentclass[10pt,fleqn,a4paper,twoside]{article}
\usepackage{abcm}
\def\shortauthor{V. Obadowski, T. Batista and P. Miyagi}
\def\shorttitle{Modeling of Naval Propulsion -- Approach Bases on Hybrid Systems}

\begin{document}
\fphead
\hspace*{-2.5mm}\begin{tabular}{||p{\textwidth}}
\begin{center}
\vspace{-4mm}
\title{MODELING OF NAVAL PROPULSION -- APPROACH BASED ON HYBRID SYSTEMS}
\end{center}
\authors{Vinícius Novicki Obadowski} \\
\authors{Thalles Andrade Estrela Batista} \\
\authors{Paulo Eigi Miyagi} \\
\institution{Escola Politécnica da Universidade de São Paulo} \\
\institution{obadowski@usp.br, thalles.batista@usp.br and pemiyagi@usp.br} \\
\\
\abstract{\textbf{Abstract.} This paper proposes a model for a full electric naval propulsion system using object-oriented differential predicate transition Petri nets (OO-DPT). This approach encompasses discrete events characteristics as well as the continuous values. To formulate this model, it was adopted the Production Flow Schema methodology in order to describe the system behavior and its main components and equipment. And after, using OO-DPT Petri Nets, a hybrid systems approach, it is possible to build a comprehensive model.}\\
\\
\keywords{\textbf{Keywords:} naval propulsion, hybrid systems, Petri Nets, Objected-oriented Differential Predicate Transition Petri Nets}\\
\end{tabular}

\section{INTRODUCTION}
\label{sec:intro}


Brazilian government has been spending resources {\it Programa de Desenvolvimento de Submarinos}  \citep{Brasil2013}

\section{NAVAL PROPULSION DESCRIPTION}
\label{sec:naval}

To write something here.

\section{MODEL}
\label{sec:model}

The Production Flow Schema (PFS) presented by \citet{Miyagi1996}\

Based on description provided by in section~\ref{sec:naval}\space, it is possible to start to model a naval propulsion

\section{ACKNOWLEDGEMENTS}
\label{sec:ack}

To write something here too.

\bibliographystyle{abcm}
\renewcommand{\refname}{}
\bibliography{library}

\section{RESPONSABILITY NOTICE}

The authors are the only responsible for the printed material included in this paper.

\end{document}
